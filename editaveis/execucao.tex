\chapter[Execução - BFS]{Execução - BFS}

Este capítulo têm como objetivo relatar o processo de desenvolvimento do algoritmo, trazendo a abordagem utilizada pelo grupo, as principais técnicas, e uma descrição do código como um todo.

\section{Descrição do Algorítmo}

Contextualizando inicialmente, têm-se duas filas: \textbf{Prioridades} e \textbf{Soluções}. Antes de iniciar o algoritmo, é feita uma preparação, onde define-se o atributo \textit{visited} da cidade origem como \textit{true}, e é inserido na fila \textbf{Prioridades} os caminhos possíveis da origem (ou seja, as fronteiras da origem). O algoritmo inicia-se com a premissa: enquanto \textbf{Prioridades} não estiver vazia, execute o bloco de código do algoritmo.

O bloco de código do algoritmo se resume a:
\begin{enumerate}
\item Pega-se o primeiro elemento de \textbf{Prioridades}, que será chamado de \textbf{aux} e é do tipo \textbf{Way}.
\item Pega-se o destino final de \textbf{Way}, irá ser chamado de \textbf{destination\textunderscore shortcut}.
\item Pega-se o caminho mais curto do \textbf{destination\textunderscore shortcut}. Irá ser chamado de \textbf{store\textunderscore path}, e é o candidato a ser expandido.
\item Os caminhos possíveis de \textbf{destination\textunderscore shortcut} precisam ser ordenados, para que o algoritmo já expanda na ordem do melhor caminho para o pior.
\item Será feito um \textit{loop interno} com \textit{n} iterações, esse \textit{n} seria a quantidade de fronteiras que \textbf{destination\textunderscore shortcut} dispõe.

  Loop interno:
  \begin{enumerate}
    \item Dentro do \textit{loop}, é visto inicialmente se o caminho proposto (\textbf{store\textunderscore path}) é uma solução para o problema. Se for, é feita uma cópia de \textbf{aux}, dentro dessa cópia é inserido o caminho dandidato, e, finalmente, \textbf{aux} é inserido em \textbf{Soluções}. Não é necessário mais continuar no \textit{loop} interno, pois uma solução já foi encontrada e todas as outras seguintes para esse destino serão inferiores.
    \item Caso não seja uma solução, é visto se o destino do candidato (\textbf{store\textunderscore path}) já foi visitado (apresenta \textit{visited} como \textit{true}). Se já tiver sido visitado, ignoramos esse candidato
    \item Caso o candidato não tenha sido visitado, finalmente o utilizamos. Inicialmente, é feita uma cópia de \textbf{aux}, por conta do problema da Persistência de Dados. Armazenamos essa cópia de \textbf{aux} nele mesmo.
    \item Dentro da nova cópia de \textbf{aux}, inserimos ao final o candidato (é dado \textit{push}).
    \item É \textit{setado} o \textit{visited} do destino do candidato como \textit{true}.
    \item Finalmente, \textbf{aux} é inserido na fila \textbf{Prioridades}.
    \item O caminho candidato agora deixa de ser \textbf{store\textunderscore path}, para ser o caminho anterior ao \textbf{store\textunderscore path}. Novamente, dizemos que \textbf{aux} é o primeiro elemento de \textbf{Prioridades}.
  \end{enumerate}

\item É dado \textit{pop} em \textbf{Prioridades}.
\end{enumerate}

\section{Contextualização}
\begin{itemize}
  \item Cidades tem Caminhos Possíveis (\textit{Paths});
  \item Caminhos Possíveis (\textit{Paths}) tem uma cidade Origem e uma cidade Destino;
  \item Um \textit{Way} é um conjunto de caminhos possíveis (\textit{Paths});
  \item Um Grafo contém a lista de Cidades;
\end{itemize}

Sendo assim, é possível inferir que as cidades nada mais são que nós do grafo, os caminhos possíveis se tratam de simples arestas do grafo, e o \textit{Way} seria o conjunto dessas arestas.

\section{Matriz de Adjacência - Não Utilizada}
Uma abordagem comum nos problemas que envolvem grafos é o da utilização da Matriz de Adjacência. É interessante pois, por conta do índice, se você sabe qual elemento deseja visitar, não é necessário percorrer nada: têm-se acesso direto ao dado.

Contudo, esta abordagem, como tudo na vida, apresenta pontos negativos. O principal é do mal uso do espaço disponível. Para ilustrar: Imagine o mesmo problema solucionado pelo grupo, mas, ao invés das 27 capitais, tendo-se 100000 municípios. Mas, independente do número de nós ter crescido muito, a média de caminhos possíveis por nós ainda continua sendo de 3 a 4. Isso significa que num vetor de 100000 espaços, 99995 não estão sendo utilizados (e isso é feito 100000 vezes). Se fosse utilizado uma abordagem de armazenamento de elo de grafo, como utilizado pelo grupo, só seria necessário armazenar os 100000 nós, o que já seria feito de qualquer jeito.

Dessa forma, é possível notar que a escolha da Matriz de Adjacência é interessante pelo acesso fácil ao dado, mas, a medida que os dados escalam (sendo assim uma grande massa de dados), o desperdício de recursos fica evidente.

\section{Descrição dos Dados Utilizados}

Os dados a serem tratados foram inseridos em arquivos de texto gerando então uma espécie de banco de dados. Cada
capital recebeu dois arquivos sendo que o primeiro possui a nomenclatura padrão ``\textit{sigla.txt}",onde ``sigla" é a sigla do estado. Cada arquivo recebe:

\begin{itemize}
 	\item Nome da capital, onde nomes compostos são separados por letras maiúsculas (padrão \textit{Capitalize});
  \item Código identificador;
	\item Número de fronteiras.
\end{itemize}

Cada campo é separado por um caracter barra vertical (|).

O segundo arquivo possue a nomenclatura ``\textit{sigla\underline{\hspace{.1in}}paths.txt}'' e eles recebem os campos:

\begin{itemize}
	\item Código da rota, que na verdade é a junção do código da capital de partida com o código da capital de destino;
	\item Nome da capital de saída (padrão \textit{Capitalize});
	\item Nome da capital de destino (padrão \textit{Capitalize});
	\item Quantidade de kilômetros do percurso.
\end{itemize}

Como no primeiro arquivo, cada campo é separado por um caracter barra vertical (|). Tais dados foram retiradas do site do Departamento Nacional de Infraestrutura de Transportes \cite{DNIT}.

\section{Modelagem - Orientação a Objetos}
A seguir, será dado um \textit{overview} a respeito de como as estruturas de dados foram tratadas no trabalho do ponto de vista do paradigma de Orientação a Objetos:
\begin{description}
  \item [Way] É um conjunto de \textbf{Paths}. Se trata de uma classe, que tem relação do tipo \textit{n para 1} com a classe \textbf{Paths}. Além disso, tem associação com a classe \textbf{City}, onde no \textit{Way} são armazenados a cidade Origem, a cidade Destino, o primeiro \textit{Path}, o ultimo \textit{Path}, a quantidade de \textit{Paths} e o tamanho da distância. É uma abstração da estrutura de dados Lista.

  \item [Path] É uma aresta do grafo, ou a distância entre uma cidade X e uma cidade Y que faz fronteira com X. Armazena uma cidade Origem, uma cidade Destino, uma distância entre as duas e um nome.

  \item [PriorityQueue] Fila de prioridades, armazena um conjunto de \textit{Ways} (relação \textit{n para 1}). Tem os atributos da cabeça do conjunto de \textit{Ways}, a calda, o total de \textit{Ways}, e uma variável extra chamada de \textbf{found}, onde é armazenado o melhor \textit{Way} até um dado momento. É uma abstração da estrutura de dados Fila.

  \item [Graph] Como dito, armazena todo o conjunto de cidades. No construtor dessa classe é feito o \textit{fetch} dos arquivos \textit{.txt}, onde são carregadas as cidades disponíveis.

  \item [City] O nó do grafo, tem os atributos nome, \textit{visited}, o primeiro \textit{Path} possível dessa cidade (ou seja, a primeira cidade com que ela faz fronteira), o ultimo \textit{Path} possível, o total de \textit{Paths} disponíveis, e um código. É uma abstração da estrutura de dados Lista.
\end{description}

\section{Relatório de Execução do Algoritmo}
