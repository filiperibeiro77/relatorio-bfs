\chapter*[Resultados Obtidos]{Resultados Obtidos}
\addcontentsline{toc}{chapter}{Resultados Obtidos}

O principal problema encontrado na execução da atividade foi a questão da persistência dos 
dados. Isso ocorria porque quando se inseria determinado elemento e, por coincidência, esse mesmo 
elemento inserido já se encontrava em uma fila diferente, a medida que novos elementos eram inseridos as referências dos elementos enfileirados eram perdidas.

Um fator que gerou uma quantidade considerável de trabalho manual e por isso demandou um grande período de 
tempo, foi a questão da inserção dos dados referentes as distâncias entre cidades dentro dos arquivos de 
texto. Não se tratava de uma tarefa de difícil abstração, porém trabalhou-se com muitos arquivos com campos 
a serem estudados. 

Outro ponto que causou certa dificuldade, foi o fato de se tratar da primeira experiência usando grafos,
que é um assunto fora do escopo da disciplina exigindo portanto vários estudos complementares para 
fundamentação dos conceitos. 

A escolha do paradigma Orientado a Objetos na resolução do problema, poderia gerar \textit{overhead}, mas essa decisão se mostrou acertada uma vez que essa abordagem abriu um leque de novas funcionalidades e 
formas de trabalhar que, caso usadas de maneira correta, facilitariam a realização da solução.

Durante a execução do trabalho, não foi possível encontrar problemas decorrentes da falta de recursos 
computacionais, nem relacionados com estouro de memória, o que demonstra que se atingiu uma solução de 
grande qualidade. Outro ponto que deve ser destacado, é que as respostas são apresentadas de forma instantânea a inserção das entradas, independente de se tratar de casos muito complexos ou não. Deve-se levar em consideração que o desenvolvimento foi executado e testado em plataformas Linux com mais de 4 GB de \textit{ram}.