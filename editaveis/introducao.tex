\chapter*[Introdução]{Introdução}
\addcontentsline{toc}{chapter}{Introdução}

Tal documento, consiste no relatório técnico da implementação do algorítmo de busca \textit{Best-First
Search}. Este algorítmo é muito utilizado quando se fala a respeito de inteligência artficial, pois com o 
mesmo pode-se usar conceitos avançados de estrutaras de dados, orientação a objetos, \textit{threads} 
entre outros. O algorítmo \textit{Best-First search}, consiste basicamente em dizer qual o possível melhor 
(menor) caminho entre dois pontos. O usuário insere o ponto inicial e o ponto que se deseja ir, com isso 
em mente o algorítmo analisa todos os possíveis caminhos e apresenta o melhor. Há variações desse 
algoritmo que permite visualizar o caminho sendo percorrido na tela.

O trabalho proposto para ser executado na disciplina consistia na implementação do mesmo, tendo como 
contexto as capitais brasileiras. Utilizando-se de dados abertos disponibilizados pelo governo, observou-se 
os caminhos que cada capital faz ligação, ou seja, quais cidades possuem uma rota direta com outras capitais. A partir disso, a solução deve permitir ao usuário que o mesmo cite uma capital qualquer, e insira o destino final do trajeto, com isso espera-se que sejam apresentadas as cidades que serão percorridas para realizar esse menor trajeto.