\chapter[Introdução]{Introdução}
\addcontentsline{toc}{chapter}{Introdução}

Independente do ramo (redes, \textit{machine learning}, inteligência artificial, detecção de padrões, labirintos, etc), um dos problemas mais comuns no contexto de grafos é em relação a localização de melhores caminhos para determinados problemas. Um exemplo comum seria o da necessidade de um GPS: o usuário se encontra em determinado lugar, que chamaremos de Origem, e deseja chegar em outro lugar, que chamaremos de Destino. O usuário deseja que o GPS seja capaz de fornecer o melhor caminho possível, levando em consideração as limitações que o próprio usuário teria a respeito do percurso (por exemplo, somente deveria ser levado em conta caminhos terrestres e não aéreos, por motivos lógicos).

Deste tipo de problema surge a necessidade de utilização de um algoritmo que seja capaz de fornecer a solução do melhor caminho. Entre os vários disponíveis (Dijkstra, Tremaux, etc), existe o BFS - Best First Search, que foi implementado e será explicado durante este relatório. No caso, o algoritmo deveria ser aplicado num contexto de capitais brasileiras, levando em consideração somente rotas rodoviárias (dados cedidos pelo DNIT).

No capítulo a seguir será dada uma explanação a respeito do algoritmo, e, no capítulo seguinte, as principais dificuldades encontradas serão levantadas. Uma observação é que, como será mostrado, o grupo utilizou o paradigma de Orientação a Objetos para apoiar o desenvolvimento do algoritmo. A escolha mostrou-se interessante por vários motivos, e será detalhada durante o relatório.

O grupo teve um enorme aprendizado no que tange Estruturas de Dados, Metaprogramação, Orientação a Objetos, Técnicas de Programação entre outros assuntos relativos a programação durante o desenvolvimento do algoritmo, e tentará passar essa ideia.