\chapter*[Descrições]{Descrições}
\addcontentsline{toc}{chapter}{Descrições}

Este capítulo têm como objetivos descrever o algorítmo e os dados que foram utilizados para resolver o problema.

\section{Descrição do Algorítmo}

O algorítmo BFS foi construído na estrutura de um grafo, onde cada capital brasileira corresponde a um nó. 
Dessa forma, o algorítmo funciona a partir da entrada de um nó (cidade), que é definido como ponto de 
partida e de um segundo nó que é o destino onde se deseja chegar. A partir disso analisa-se os caminhos possíveis que o nó pode percorrer, começando pelo de menor caminho, depois o de segundo menor caminho e assim sucessivamente. Cada análise de nó pode ser entendida como um novo enfileiramento. A cada iteração com um nó vizinho o tamanho do caminho é somado a fim de se descobrir qual o menor deles. Quando não há mais subcaminhos a serem estudados, pois o destino do mesmo é um ponto que já teve sua análise feita, é realizado o desempilhamento de tal subcaminho já que o mesmo se mostra inválido para a solução.

\section{Descrição dos Dados Utilizados}

Os dados a serem tratados foram inseridos em arquivos de texto gerando então um banco de dados. Cada 
capital recebeu dois arquivos sendo que o primeiro possue a nomenclatura padrão ``\textit{sigla.txt}",onde ``sigla" é a sigla do estado. Cada arquivo recebe: 

\begin{itemize}
 	\item Nome da capital, onde nomes compostos são separados por letras maiúsculas (padrão \textit{CamelCase});
    \item Código identificador;
	\item Número de capitais que possuem rota direta. 
\end{itemize}

Cada campo é separado por um caracter barra vertical (|). 

O segundo arquivo possue a nomenclatura ``\textit{sigla\underline{\hspace{.1in}}paths.txt}'' e eles recebem os campos:

\begin{itemize}
	\item Código da rota, que na verdade é a junção do código da capital de partida com o código da capital de destino;
	\item Nome da capital de saída (padrão \textit{CamelCase});
	\item Nome da capital de destino (padrão \textit{CamelCase});
	\item Quantidade de kilômetros do percurso.
\end{itemize}

Como no primeiro arquivo, cada campo é sepadado por um caracter barra vertical (|). Tais dados foram retiradas do site do Departamento Nacional de Infraestrutura de Transportes \cite{DNIT}.
