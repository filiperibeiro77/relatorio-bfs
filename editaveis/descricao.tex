\chapter*[Descrições]{Descrições}
\addcontentsline{toc}{chapter}{Descrições}

Este capítulo têm como objetivos descrever o algorítmo e os dados que foram utilizados para resolver o problema.

\section{Descrição do Algorítmo}

O algorítmo BFS foi construído na estrutura de um grafo, onde cada capital brasileira corresponde a um nó. 
Com isso em mente, o algorítmo funciona a partir da entrada de um nó (cidade), que é definida como ponto de 
partida e de um segundo nó que é o destino onde se deseja chegar. A partir disso, começa-se a analisar 
todos os pontos que possuem rota direta com o ponto atual, e opta-se por escolher o menor deles como 
caminho. Essa lógica se repete até o momento em que se chega ao ponto desejado. Porém nem sempre o caminho 
é composto apenas pelas menores distâncias, por isso, há uma análise em paralelo dos outros nós, e o melhor 
caminho é escolhido a partir da menor soma total de distância dos caminhos que se obteve.

Algo a ser citado, é que os nós nunca podem ser revisitados, pois caso isso aconteça, o algorítmo entra em 
um \textit{loop} infinito, uma vez que sempre em algum ponto, a menor distância será um nó já visitado. 
Para esse tratamento, usou-se um atributo que faz essa verificação. 

\section{Descrição dos Dados Utilizados}

Os dados a serem tratados, foram inseridos em arquivos de texto, gerando então um banco de dados. Cada 
capital, recebeu dois arquivos, sendo que os primeiros possuem a nomenclatura padrão ``\textit{sigla.txt}'' onde ``sigla", é a sigla do estado.
e recebem: 

\begin{itemize}
 	\item Nome da capital, onde nomes compostos são separados por letras maiúsculas (padrão \textit{CamelCase});
    \item Código identificador;
	\item Número de capitais que possuem rota direta. 
\end{itemize}

Cada campo, é sepadado por um caracter barra vertical (|). 

Os segundos arquivos, possuem a nomenclatura ``\textit{sigla\underline{\hspace{.1in}}paths.txt}'' e eles recebem os campos:

\begin{itemize}
	\item Código da rota, que na verdade é a junção do código da capital de partida com o código da capital de destino;
	\item Nome da capital de saída (padrão \textit{CamelCase});
	\item Nome da capital de destino (padrão \textit{CamelCase});
	\item Quantidade de kilômetros do percurso.
\end{itemize}

Como no primeiro arquivo, cada campo, é sepadado por um caracter barra vertical (|). Tais dados foram retiradas do site do Departamento Nacional de Infraestrutura de Transportes \cite{DNIT}
